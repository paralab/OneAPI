\documentclass[]{report}   % list options between brackets
\usepackage{}              % list packages between braces
\usepackage{pgfplots}
\usepackage{float}
% type user-defined commands here

\begin{document}

\title{Evaluation of Intel One API Beta Version}   % type title between braces
\author{Milinda Fernando}         % type author(s) between braces
\date{June 12, 2020}    % type date between braces
\maketitle

\begin{abstract}
  This document presents performance evaluation of Intel's OneAPI in heterogeneous compute environment. We focus on three kernel, vector dot product (BLAS level 1), matrix-vector (BLAS level 2) and matrix-matrix (BLAS level 3) product using Intel's MKL implementation for GPU and CPUs. Currently, OneAPI beta version support is limited to Intel GPUs only. 
\end{abstract}

\section{Experimental Setup}
Presented experiments were conducted on Intel's developer cloud facility. 


\begin{verbatim}
Architecture:        x86_64
CPU op-mode(s):      32-bit, 64-bit
Byte Order:          Little Endian
CPU(s):              12
On-line CPU(s) list: 0-11
Thread(s) per core:  2
Core(s) per socket:  6
Socket(s):           1
NUMA node(s):        1
Vendor ID:           GenuineIntel
CPU family:          6
Model:               158
Model name:          Intel(R) Xeon(R) E-2176G CPU @ 3.70GHz
Stepping:            10
CPU MHz:             1100.033
CPU max MHz:         4000.0000
CPU min MHz:         800.0000
BogoMIPS:            7392.00
Virtualization:      VT-x
L1d cache:           32K
L1i cache:           32K
L2 cache:            256K
L3 cache:            12288K
NUMA node0 CPU(s):   0-11
\end{verbatim}  

From Sycl, this is the GPU info used at the runtime. 

\begin{verbatim}

Device: Intel(R) Gen9 HD Graphics NEO
global mem: 53882437632
local  mem: 65536

\end{verbatim}

\newpage
\section{Results}

\begin{figure}[H]
	\centering
	\begin{tikzpicture}
	\begin{axis}[
	ybar,
	symbolic x coords={1M,10M,100M},
	xtick=data,
	nodes near coords,
	nodes near coords align={vertical},
	legend pos= outer north east,
	width =8cm,height=5cm
	]
	\addplot [fill=blue!50, fill opacity=0.5] [bar shift=-.2cm] table[x={inSz}, y = {mkl_seq}]{dat/blas_1.dat};
	\addplot [fill=red!50, fill opacity=0.5] [bar shift=.2cm] table[x={inSz}, y = {dev(s)}]{dat/blas_1.dat};
	\legend{MKL(CPU),MKL(GPU)}
	\end{axis}
	\end{tikzpicture}
	\caption{Performance comparison of Intel's OneMKL for BLAS 1 dot product routine}
\end{figure}


\begin{figure}[H]
	\centering
	\begin{tikzpicture}
	\begin{axis}[
	ybar,
	symbolic x coords={1K,5K,10K,20K},
	xtick=data,
	nodes near coords,
	nodes near coords align={vertical},
	legend pos= outer north east,
	width =8cm, height=5cm
	]
	\addplot [fill=blue!50, fill opacity=0.5] [bar shift=-.2cm] table[x={inSz}, y = {mkl_seq}]{dat/blas_2.dat};
	\addplot [fill=red!50, fill opacity=0.5] [bar shift=.2cm] table[x={inSz}, y = {dev(s)}]{dat/blas_2.dat};
	\legend{MKL(CPU),MKL(GPU)}
	\end{axis}
	\end{tikzpicture}
	\caption{Performance comparison of Intel's OneMKL for BLAS 2 dense mat-vec product routine}
\end{figure}



\begin{figure}[H]
	\centering
	\begin{tikzpicture}
	\begin{axis}[
	ybar,
	symbolic x coords={1K,2K,3K,3.5K},
	xtick=data,
	nodes near coords,
	nodes near coords align={vertical},
	legend pos= outer north east,
	width =8cm,height=5cm
	]
	\addplot [fill=blue!50, fill opacity=0.5] [bar shift=-.2cm] table[x={inSz}, y = {mkl_seq}]{dat/blas_3.dat};
	\addplot [fill=red!50, fill opacity=0.5] [bar shift=.2cm] table[x={inSz}, y = {dev(s)}]{dat/blas_3.dat};
	\legend{MKL(CPU),MKL(GPU)}
	\end{axis}
	\end{tikzpicture}
	\caption{Performance comparison of Intel's OneMKL for BLAS 3 dense matrix-matrix product routine}
\end{figure}



%\begin{thebibliography}{9}
  % type bibliography here
%\end{thebibliography}

\end{document}